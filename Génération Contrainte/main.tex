\documentclass[9pt]{article}

% This assumes your files are encoded as UTF8
\usepackage[utf8]{inputenc}
\usepackage[T1]{fontenc}
% It will improve the aesthetics of text in the typewriter font.
\usepackage{graphicx,float}
\usepackage{natbib}
\usepackage{glossaries}

\makeglossaries
% Définitions pour le glossaire
\newglossaryentry{AutoregressiveStructuredPrediction}{
    name=Prédiction Structurée Autoregressive,
    description={Une technique où chaque élément de la sortie est prédit en fonction des éléments précédemment générés, utile pour des tâches comme l'analyse syntaxique et l'extraction d'informations.}
}


\newglossaryentry{ConstrainedGeneration}{
    name=Génération Contrainte,
    description={}
}

\newglossaryentry{ConstrainedDecoding}{
    name=Décodage Contraint,
    description={Une méthode pour guider la génération de texte en appliquant des contraintes sur la sortie, telles que le format, la validité de la séquence de tokens, ou le respect de certaines règles comme la grammaire.}
}

\newglossaryentry{FST}{
    name=FST (Transducteurs à États Finis),
    description={Un modèle computationnel utilisé pour représenter les transformations entre deux ensembles de symboles, souvent appliqué à la normalisation de texte ou aux tâches linguistiques.}
}

\newglossaryentry{FSA}{
    name=FSA (Automates à États Finis),
    description={Un modèle mathématique qui décrit un système avec un nombre fini d'états. Il est utilisé pour imposer des contraintes sur les séquences générées, garantissant la validité des séquences de tokens ou de mots.}
}

\newglossaryentry{PDA}{
    name=PDA (Automates à Pile Déterministes),
    description={Un modèle computationnel qui utilise une pile pour stocker des données, utilisé pour appliquer des contraintes complexes comme les grammaires indépendantes du contexte.}
}

\newglossaryentry{LexicalConstraints}{
    name=Contraintes Lexicales,
    description={Des limitations appliquées au vocabulaire qu'un modèle peut utiliser lors de la génération, garantissant que seuls certains mots ou tokens sont produits.}
}

\newglossaryentry{GrammaticalConstraints}{
    name=Contraintes Grammaticales,
    description={Des règles qui restreignent la sortie pour qu'elle soit grammaticalement correcte, souvent appliquées à l'aide d'automates ou de grammaires formelles lors des tâches de génération de texte.}
}

\newglossaryentry{SLU}{
    name=SLU (Compréhension du Langage Parlé),
    description={Une branche du traitement du langage naturel (NLP) qui se concentre sur la compréhension du langage parlé par des machines, impliquant généralement des tâches comme le remplissage d'emplacements et la classification d'intentions.}
}

\newglossaryentry{Tokenization}{
    name=Tokenisation,
    description={Le processus de division d'une chaîne de texte en tokens individuels (mots, sous-mots ou caractères) pour le traitement dans les tâches de NLP.}
}

\newglossaryentry{SemanticParsing}{
    name=Analyse Sémantique,
    description={Le processus de conversion d'une phrase en langage naturel en une représentation formelle, telle qu'une forme logique ou du code, généralement contrainte par des règles grammaticales.}
}

\newglossaryentry{EntityDisambiguation}{
    name=Désambiguïsation d'Entités,
    description={Une tâche où un modèle identifie et relie des entités ambiguës (par exemple, des noms) à leur référence correcte dans une base de connaissances.}
}

\newglossaryentry{EntityLinking}{
    name=Liaison d'Entités,
    description={Une tâche dans le NLP où les mentions d'entités dans un texte sont reliées à leurs entrées correspondantes dans une base de connaissances, telle que Wikipedia.}
}

\newglossaryentry{SRL}{
    name=SRL (Annotation de Rôles Sémantiques),
    description={Le processus d'identification et d'étiquetage des rôles des divers constituants d'une phrase par rapport à un prédicat, tels que le sujet, l'objet ou d'autres rôles grammaticaux.}
}

\newglossaryentry{DSL}{
    name=DSL (Langage Spécifique au Domaine),
    description={Un langage spécialisé utilisé pour exprimer des concepts dans un domaine spécifique, souvent employé dans des tâches comme l'analyse sémantique ou la génération de code.}
}

\newglossaryentry{Trie}{
    name=Structures de Données Basées sur Trie,
    description={Une structure de données en forme d'arbre utilisée pour stocker et rechercher efficacement des séquences de tokens, couramment utilisée pour les contraintes lexicales lors de la génération.}
}

\newglossaryentry{BeamSearch}{
    name=Recherche en Faisceau,
    description={Un algorithme de recherche heuristique utilisé dans le décodage contraint pour explorer plusieurs chemins en parallèle, améliorant ainsi l'efficacité en élaguant les chemins moins prometteurs.}
}

\newglossaryentry{ConstituencyParsing}{
    name=Analyse en Constituants,
    description={Un type d'analyse syntaxique qui divise une phrase en une structure hiérarchique, souvent visualisée sous forme d'arbre, montrant les constituants grammaticaux.}
}

\newglossaryentry{SMILES}{
    name=SMILES (Simplified Molecular Input Line Entry System),
    description={Une notation qui permet à un utilisateur de représenter une structure chimique d'une manière compréhensible par les ordinateurs, souvent contrainte dans les tâches de génération de molécules.}
}


\providecommand{\tightlist}{%
  \setlength{\itemsep}{0pt}\setlength{\parskip}{0pt}}

\title{Constrained Generation \& AR Structured Predictions for NLU Tasks}
% \title{Génération structurée/contrainte avec LLMs}

\begin{document}

\maketitle

\section{Introduction}

Dans ce document, nous ferons une revue de littérature sur la \gls{ConstrainedGeneration}. 

% Nous commencerons par une revue des tâches de NLP puis nous étudierons les méthodes de \gls{ConstrainedGeneration}. Enfin, nous nous intéresserons aux bibliothèques (python) implémentant la \gls{ConstrainedDecoding}.

% Nous discuterons ensuite des tâches de \gls{SLU} et des méthodes de \gls{ConstrainedDecoding} utilisées dans ces tâches. Nous finirons par une revue des systèmes actuels de \gls{SLU} et des méthodes de \gls{ConstrainedDecoding} utilisées dans ces systèmes.

% \section{Constat sur les tâches de NLP}
% Les tâches de NLP sont structurées.

% \begin{enumerate}
%     \tightlist
%     \item Classification
%     \item NER
% \end{enumerate}

% \begin{figure}
%     \includegraphics[width=\textwidth]{assets/Example_Slotfilling.pdf}
%     \caption{Exemple des tâches de \gls{SLU}}
% \end{figure}


\section{Différents Types de Contraintes}\label{differents-types-de-contraintes}

On ditingue des contraintes de différents niveaux :

\begin{itemize}
\tightlist
\item Contrainte de domaine
\item Contrainte de tâches
\item Contrainte de vocabulaire
\end{itemize}

\subsection{Contraintes de domaine}\label{contraintes-de-domaine}

Les modèles doivent souvent s'adapter à un contexte ou à un domaine particulier, tel que la médecine, la finance ou le droit. 
Les contraintes de domaine visent à restreindre la génération de séquences à un style spécifique à ce domaine, permettant au modèle de produire des sorties plus adaptées et pertinentes. 

Exemple : dans la génération de texte dans le domaine bancaire, le modèle doit respecter des formats, termes bancaires et abréviations spécifiques au domaine.

\subsection{Contraintes de tâche}\label{contraintes-de-tuxe2che}

Une contrainte de tâche se réfère à la génération de texte pour résoudre une tâche donnée/précise. 
Cela peut inclure des contraintes liées à la syntaxe ou à la structure (exemple : formats JSON, XML, balisage HTML) ou à des sorties spécifiques pour une tâche de classification ou de génération. 
On peut considerer que c'est une sorte un regroupement des deux types de contraintes précédentes. 

Exemple : produire un résumé en respectant une longueur maximale (par exemple 200 caractères).

\subsection{Contraintes de vocabulaire}{Contraintes de vocabulaire}\label{contraintes-de-vocabulaire}

Ici, on impose un sous-ensemble du vocabulaire autorisé à un modèle. 
Cela peut être utile dans des tâches de traduction, d'écriture de code ou d'autres scénarios où certains termes sont invalides ou, au contraire, doivent obligatoirement apparaître dans la sortie. 

Exemple : dans une tâche de génération de code, le modèle doit limiter ses prédictions aux syntaxes valides pour un langage de programmation spécifique.

\section{Phases d'application des contraintes}\label{phases-dapplication-des-contraintes}

On peut contraindre un modèle statistique à différentes phases
d'entrainement ou d'utilisation.

\begin{itemize}
    \tightlist
    \item Contraintes durant l'entrainement/l'apprentissage
    \item Contraintes durant l'inférence
\end{itemize}

\subsection{Contraintes durant l'entraînement}\label{contraintes-durant-lentrauxeenement}

\begin{enumerate}
    \def\labelenumi{\arabic{enumi}.}
    \tightlist
    \item Apprentissage général pour suivre des contraintes (Instruction-tuning, RLHF/AI).
    \item Apprentissage multi-tâches avec des tokens/classes pour chaque tâches (apprendre au modèle à résoudre plusieurs tâches).
    \item Apprentissage sur la tâche directement (Cas particulier de 2. - apprendre au modèle à résoudre la tâche).
\end{enumerate}

\subsection{Contraintes durant l'inférence}\label{contraintes-durant-linfuxe9rence}

Durant l'inference on cherche à ce que le modèle prédise des séquences qui respectent des contraintes, on va donc modifier le support de décodage pour que le modèle produise des séquences qui respectent les contraintes.
Modifier le support de décodage, c'est à dire repondérer les scores (logits, probabilités) des tokens en fonction des contraintes.
On dit aussi que l'on biaise les prédictions du modèle.

On peut le faire de manière brutale (en forçant les scores de certains tokens à 0) ou de manière plus douce (en reduisant les scores de ces tokens). 

\begin{itemize}
    \tightlist
    \item Utiliser/modifier les paramètres de décodage: Top-K, Top-p, greedy search, beam-search\cite{anderson2017guidedopenvocabularyimage,post2018fastlexicallyconstraineddecoding,hu-etal-2019-improved,li2021guidedgenerationcauseeffect}, sampling, etc.
    \item Cooperative decoding: on pondère les scores de chacun des tokens en fonction d'un système externe (discriminateur\cite{krause2020gedigenerativediscriminatorguided,dathathri2020plugplaylanguagemodels}, modèle de langue\cite{liu2024tuninglanguagemodelsproxy,Shi2024DecodingTimeLM,Fan2024OnGS}, Automates\cite{})
    \item Stratégies de Prompting: Direct Answer, Chain of Thoughts (CoT), self-consistency/verification, etc.
\end{itemize}

On peut aussi combiner les différentes méthodes pour obtenir des résultats plus performants. 
Par exemple, utiliser un prompt particulier + utiliser un discriminateur externe pour ranker les tokens. 

\input{decoding_parameters}

\input{cooperative_decoding}

\input{prompting_strategies}

\newpage

\section{References}

\bibliographystyle{plainnat}
\bibliography{custom,Constrained_Decoding,cooperative_decoding,COLM}

\newpage
\section{Glossaire}
% Print the glossary
\printglossaries

\end{document}
