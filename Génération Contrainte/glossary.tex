% Définitions pour le glossaire
\newglossaryentry{AutoregressiveStructuredPrediction}{
    name=Prédiction Structurée Autoregressive,
    description={Une technique où chaque élément de la sortie est prédit en fonction des éléments précédemment générés, utile pour des tâches comme l'analyse syntaxique et l'extraction d'informations.}
}


\newglossaryentry{ConstrainedGeneration}{
    name=Génération Contrainte,
    description={}
}

\newglossaryentry{ConstrainedDecoding}{
    name=Décodage Contraint,
    description={Une méthode pour guider la génération de texte en appliquant des contraintes sur la sortie, telles que le format, la validité de la séquence de tokens, ou le respect de certaines règles comme la grammaire.}
}

\newglossaryentry{FST}{
    name=FST (Transducteurs à États Finis),
    description={Un modèle computationnel utilisé pour représenter les transformations entre deux ensembles de symboles, souvent appliqué à la normalisation de texte ou aux tâches linguistiques.}
}

\newglossaryentry{FSA}{
    name=FSA (Automates à États Finis),
    description={Un modèle mathématique qui décrit un système avec un nombre fini d'états. Il est utilisé pour imposer des contraintes sur les séquences générées, garantissant la validité des séquences de tokens ou de mots.}
}

\newglossaryentry{PDA}{
    name=PDA (Automates à Pile Déterministes),
    description={Un modèle computationnel qui utilise une pile pour stocker des données, utilisé pour appliquer des contraintes complexes comme les grammaires indépendantes du contexte.}
}

\newglossaryentry{LexicalConstraints}{
    name=Contraintes Lexicales,
    description={Des limitations appliquées au vocabulaire qu'un modèle peut utiliser lors de la génération, garantissant que seuls certains mots ou tokens sont produits.}
}

\newglossaryentry{GrammaticalConstraints}{
    name=Contraintes Grammaticales,
    description={Des règles qui restreignent la sortie pour qu'elle soit grammaticalement correcte, souvent appliquées à l'aide d'automates ou de grammaires formelles lors des tâches de génération de texte.}
}

\newglossaryentry{SLU}{
    name=SLU (Compréhension du Langage Parlé),
    description={Une branche du traitement du langage naturel (NLP) qui se concentre sur la compréhension du langage parlé par des machines, impliquant généralement des tâches comme le remplissage d'emplacements et la classification d'intentions.}
}

\newglossaryentry{Tokenization}{
    name=Tokenisation,
    description={Le processus de division d'une chaîne de texte en tokens individuels (mots, sous-mots ou caractères) pour le traitement dans les tâches de NLP.}
}

\newglossaryentry{SemanticParsing}{
    name=Analyse Sémantique,
    description={Le processus de conversion d'une phrase en langage naturel en une représentation formelle, telle qu'une forme logique ou du code, généralement contrainte par des règles grammaticales.}
}

\newglossaryentry{EntityDisambiguation}{
    name=Désambiguïsation d'Entités,
    description={Une tâche où un modèle identifie et relie des entités ambiguës (par exemple, des noms) à leur référence correcte dans une base de connaissances.}
}

\newglossaryentry{EntityLinking}{
    name=Liaison d'Entités,
    description={Une tâche dans le NLP où les mentions d'entités dans un texte sont reliées à leurs entrées correspondantes dans une base de connaissances, telle que Wikipedia.}
}

\newglossaryentry{SRL}{
    name=SRL (Annotation de Rôles Sémantiques),
    description={Le processus d'identification et d'étiquetage des rôles des divers constituants d'une phrase par rapport à un prédicat, tels que le sujet, l'objet ou d'autres rôles grammaticaux.}
}

\newglossaryentry{DSL}{
    name=DSL (Langage Spécifique au Domaine),
    description={Un langage spécialisé utilisé pour exprimer des concepts dans un domaine spécifique, souvent employé dans des tâches comme l'analyse sémantique ou la génération de code.}
}

\newglossaryentry{Trie}{
    name=Structures de Données Basées sur Trie,
    description={Une structure de données en forme d'arbre utilisée pour stocker et rechercher efficacement des séquences de tokens, couramment utilisée pour les contraintes lexicales lors de la génération.}
}

\newglossaryentry{BeamSearch}{
    name=Recherche en Faisceau,
    description={Un algorithme de recherche heuristique utilisé dans le décodage contraint pour explorer plusieurs chemins en parallèle, améliorant ainsi l'efficacité en élaguant les chemins moins prometteurs.}
}

\newglossaryentry{ConstituencyParsing}{
    name=Analyse en Constituants,
    description={Un type d'analyse syntaxique qui divise une phrase en une structure hiérarchique, souvent visualisée sous forme d'arbre, montrant les constituants grammaticaux.}
}

\newglossaryentry{SMILES}{
    name=SMILES (Simplified Molecular Input Line Entry System),
    description={Une notation qui permet à un utilisateur de représenter une structure chimique d'une manière compréhensible par les ordinateurs, souvent contrainte dans les tâches de génération de molécules.}
}
